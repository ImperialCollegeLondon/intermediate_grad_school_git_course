\documentclass[9pt]{extarticle}
\usepackage{fullpage}
\usepackage{tcolorbox}
\usepackage[colorlinks=true, urlcolor=blue, linkcolor=red]{hyperref}

\definecolor{navy}{RGB}{0,33,71}
\definecolor{lightgrey}{RGB}{235,238,238}
\definecolor{imperialblue}{RGB}{0,0,205}
\definecolor{lightblue}{RGB}{212,239,252}
\definecolor{coolgrey}{RGB}{157,157,157}
\setlength{\parindent}{0pt}

\newcommand{\boxed}[4][]{
  \begin{tcolorbox}[width=(0.95\textwidth)/#2,title=#3,before=,after=\hfill,
      colback=lightgrey,colframe=imperialblue,fonttitle=\bfseries,#1]
    #4
  \end{tcolorbox}
}
\newenvironment{navybox}[1][]{
  \begin{tcolorbox}[colback=lightgrey,colframe=navy,
      fonttitle=\bfseries,#1]
}{
  \end{tcolorbox}
}
\newenvironment{bluebox}[1][]{
  \begin{tcolorbox}[colback=lightgrey,colframe=imperialblue,
      fonttitle=\bfseries,#1]
}{
  \end{tcolorbox}
}

\newcommand{\furtherhelp}{
\subsection*{Further Help}

This course was developed by the \href{https://www.imperial.ac.uk/admin-services/ict/self-service/research-support/rcs/}{Research Computing Service (RCS)}
at Imperial College London, in particular by the Research Software Engineering (RSE)
team.

The RSE team are a part of Imperial ICT combining specialist knowledge in software
engineering with extensive experience in research. The team works with academic groups
on a wide range of projects whilst also organising community events and training (such
as this course) for the benefit of the research community. You can find out more at
\href{https://www.imperial.ac.uk/admin-services/ict/self-service/research-support/rcs/service-offering/research-software-engineering/}{the RSE team website}
and \href{https://www.imperial.ac.uk/computational-methods/rse/}{the Imperial Research
Software Community website}.
You can also consult the expertise of the RSE team by \href{https://www.imperial.ac.uk/admin-services/ict/self-service/research-support/rcs/service-offering/research-software-engineering/code-surgeries/}{booking a code surgery appointment}.

You may also be interested in attending another course developed by the RSE team titled
\textbf{``Essential Software Engineering for Researchers''} which can be
\href{https://www.imperial.ac.uk/students/academic-support/graduate-school/professional-development/doctoral-students/research-computing-data-science/courses/essential-software-engineering-for-researchers/}{booked via the Graduate School}.
}

\usepackage{multicol}

\def\boxpad{4pt}
\tcbsetforeverylayer{left=\boxpad,right=\boxpad,top=\boxpad,bottom=\boxpad}

\def\itempad{-1pt}


\begin{document}
\thispagestyle{empty}

\begin{center}
  {\LARGE
  Git/GitHub cheatsheet}
\end{center}

\begin{bluebox}[title=Things to think when publishing code]
  \begin{itemize}
    \itemsep\itempad
  \item \textbf{Public or Private?} - What level of visibility should your
    project to have? Open code promotes reproducibility and impact but be sure
    to consider and consult any other contributors
  \item \textbf{README.md} - The `front page' of a repository. GitHub will
    automatically render the markdown in this file nicely
  \item \textbf{Licence} - Tell other people what they can and cannot do with
    your code. The College recommended licence is BSD 2- or 3-clause
  \item \textbf{INSTALL.md} - If your project is complex to setup and install be
    sure to include detailed instructions
  \item \textbf{CITATION.cff} - Show how your project should be cited by others
  \item \textbf{CONTRIBUTING.md} - Tell potential contributors the kind of work
    you're interested in and the process they should follow
  \end{itemize}
\end{bluebox}

\begin{bluebox}[title=Glossary]
  \begin{itemize}
    \itemsep\itempad
    \item \textbf{Repository/Repo} -- A project that is being managed by Git
    \item \textbf{Commit} -- A set of changes recorded in the history of
      a project
    \item \textbf{Staging Area} -- The location where file changes are recorded to
      prepare for inclusion in a commit
    \item \textbf{Working Tree} -- The visible copy of files in a project that you view
      and edit as usual
    \item \textbf{Branch} -- A label used for a set of commits with a particular
      purpose
    \item \textbf{Merge} -- Combine the changes from another branch (or commit etc.)
      into the current one
    \item \textbf{Merge Conflict} -- A state that occurs when merging automatically
      fails because two sets of changes are incompatible. In this case, the changes must
      be resolved manually.
    \item \textbf{main} -- The default name given to the starting branch of all
      repositories
    \item \textbf{HEAD} -- The current commit that the working tree is based on
    \item \textbf{Local} -- A repository copy stored on the computer where you are
      working
    \item \textbf{Remote} -- A repository copy stored elsewhere, usually a hosting
      service like GitHub
    \item \textbf{origin} -- The default name given to a repository's remote
    \item \textbf{Upstream} -- The remote repository with which a local repository is
      associated
    \item \textbf{Tracking} -- Used to describe a branch in a local repository which is
      matched with a branch in a remote repository
    \item \textbf{Fork} -- A copy of a repository on GitHub that is owned by a different
      GitHub user
    \item \textbf{Origin} -- The default name used by Git for a configured remote
      repository
    \item \textbf{Pull Request (PR)} -- A GitHub feature which requests that changes from a
      fork be incorporated into the original repository
    \item \textbf{Continuous Integration (CI)} -- A software development practice which
      involves running automated checks to ensure code contributions meet certain
      criteria. An example of a CI system is GitHub Actions.
    \item \textbf{Semantic Versioning} -- A versioning scheme where the different
      numbers in a software version (e.g. v1.2.3) have a particular meaning
  \end{itemize}
\end{bluebox}

(Continued overleaf.)

\begin{bluebox}[title=Git Command Cheat Sheet]
  \begin{itemize}
    \itemsep\itempad
  \item \textbf{git config} -- Change (or view) the settings that Git uses
  \item \textbf{git init} -- Create a new repository in the current directory
  \item \textbf{git status} -- High-level overview of changes made since the last commit
  \item \textbf{git stage} -- Stage a file (or changes made to a file) for inclusion in
    the next commit
  \item \textbf{git add} -- See ``git stage''
  \item \textbf{git commit -m ``COMMIT MESSAGE''} -- Create a new commit including all
    staged file changes
  \item \textbf{git commit -{}-amend} -- Incorporate further changes into the last commit
    and/or edit the commit message
  \item \textbf{git log} -- Display the commit history of a repository
  \item \textbf{git diff} -- Show in detail the changes made in the working directory
    since the last commit
  \item \textbf{git reset -{}-soft HEAD\^} -- Remove the last commit from the history of
    a repository
  \item \textbf{git revert -{}-no-edit COMMIT\_HASH} -- Create a new commit that undoes
    the changes of the specified commit
  \item \textbf{git branch} -- Report on the existing branches in a repository
  \item \textbf{git branch BRANCH\_NAME} -- Create a new branch
  \item \textbf{git branch -D BRANCH\_NAME} -- Delete a branch (be careful with this one!)
  \item \textbf{git switch BRANCH\_NAME} -- Update the position of HEAD to
    a new branch
  \item \textbf{git switch -{}-detach COMMIT\_NAME} -- Update the position of HEAD to a
    new commit
  \item \textbf{git checkout} -- See ``git switch''
  \item \textbf{git merge -{}-no-edit BRANCH\_NAME} -- Merge BRANCH\_NAME into the
    current branch
  \item \textbf{git rebase NEW\_BASE} -- Rebase current branch onto NEW\_BASE
  \item \textbf{git remote add origin REPOSITORY\_URL} -- Configure a local repository
    with a remote with the label `origin'
  \item \textbf{git push} -- Synchronise changes in the current local branch to its
    upstream branch
  \item \textbf{git push -{}-tags} -- As above, but also push any tags you've created to
    the remote
  \item \textbf{git pull} -- Synchronise changes in the upstream branch to the current
    local one
  \item \textbf{git clone REPOSITORY\_URL} -- Create a new local repository that is a
    copy of remote one
  \item \textbf{git tag TAG\_NAME [COMMIT\_HASH]} -- Create a new tag at the specified
    commit (or at HEAD, if not specified)
  \end{itemize}
\end{bluebox}

For a more exhaustive description of the various Git commands and their options, you can
consult \href{https://git-scm.com/docs}{Git's online documentation}.


\furtherhelp

\end{document}
