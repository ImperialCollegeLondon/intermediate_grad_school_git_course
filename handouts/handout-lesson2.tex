\documentclass[10pt]{extarticle}
\usepackage{fullpage}
\usepackage{tcolorbox}
\usepackage[colorlinks=true, urlcolor=blue, linkcolor=red]{hyperref}

\definecolor{navy}{RGB}{0,33,71}
\definecolor{lightgrey}{RGB}{235,238,238}
\definecolor{imperialblue}{RGB}{0,0,205}
\definecolor{lightblue}{RGB}{212,239,252}
\definecolor{coolgrey}{RGB}{157,157,157}
\setlength{\parindent}{0pt}

\newcommand{\boxed}[4][]{
  \begin{tcolorbox}[width=(0.95\textwidth)/#2,title=#3,before=,after=\hfill,
      colback=lightgrey,colframe=imperialblue,fonttitle=\bfseries,#1]
    #4
  \end{tcolorbox}
}
\newenvironment{navybox}[1][]{
  \begin{tcolorbox}[colback=lightgrey,colframe=navy,
      fonttitle=\bfseries,#1]
}{
  \end{tcolorbox}
}
\newenvironment{bluebox}[1][]{
  \begin{tcolorbox}[colback=lightgrey,colframe=imperialblue,
      fonttitle=\bfseries,#1]
}{
  \end{tcolorbox}
}

\newcommand{\furtherhelp}{
\subsection*{Further Help}

This course was developed by the \href{https://www.imperial.ac.uk/admin-services/ict/self-service/research-support/rcs/}{Research Computing Service (RCS)}
at Imperial College London, in particular by the Research Software Engineering (RSE)
team.

The RSE team are a part of Imperial ICT combining specialist knowledge in software
engineering with extensive experience in research. The team works with academic groups
on a wide range of projects whilst also organising community events and training (such
as this course) for the benefit of the research community. You can find out more at
\href{https://www.imperial.ac.uk/admin-services/ict/self-service/research-support/rcs/service-offering/research-software-engineering/}{the RSE team website}
and \href{https://www.imperial.ac.uk/computational-methods/rse/}{the Imperial Research
Software Community website}.
You can also consult the expertise of the RSE team by \href{https://www.imperial.ac.uk/admin-services/ict/self-service/research-support/rcs/service-offering/research-software-engineering/code-surgeries/}{booking a code surgery appointment}.

You may also be interested in attending another course developed by the RSE team titled
\textbf{``Essential Software Engineering for Researchers''} which can be
\href{https://www.imperial.ac.uk/students/academic-support/graduate-school/professional-development/doctoral-students/research-computing-data-science/courses/essential-software-engineering-for-researchers/}{booked via the Graduate School}.
}


\def\boxpad{4pt}
\tcbsetforeverylayer{left=\boxpad,right=\boxpad,top=\boxpad,bottom=\boxpad}

\def\itempad{0pt}


\begin{document}
\thispagestyle{empty}

\begin{center}
  {\Large
  Using Git to Code, Collaborate and Share - Lesson 2 - URL: v.gd/gitcourse}
\end{center}

\begin{bluebox}[title=Things to think when publishing code]
  \begin{itemize}
    \itemsep\itempad
  \item \textbf{Public or Private?} - What level of visibility should your
    project to have? Open code promotes reproducibility and impact but be sure
    to consider and consult any other contributors
  \item \textbf{README.md} - The `front page' of a repository. GitHub will
    automatically render the markdown in this file nicely
  \item \textbf{Licence} - Tell other people what they can and cannot do with
    your code. The College recommended licence is BSD 2- or 3-clause
  \item \textbf{INSTALL.md} - If your project is complex to setup and install be
    sure to include detailed instructions
  \item \textbf{CITATION.txt} - Show how your project should be cited by others
  \item \textbf{CONTRIBUTING.md} - Tell potential contributors the kind of work
    you're interested in and the process they should follow
  \end{itemize}
\end{bluebox}

\begin{bluebox}[title=Git Glossary]
  \begin{itemize}
      \itemsep\itempad
    \item \textbf{Local} - A repository copy stored on the computer where you
      are working
    \item \textbf{Remote} - A repository copy stored elsewhere, usually a
      hosting service like GitHub
    \item \textbf{Upstream} - The remote repository with which a local
      repository is associated
    \item \textbf{Tracking} - Used to describe a branch in a local repository
      which is matched with a branch in a remote repository
    \item \textbf{Fork} - A copy of a repository on GitHub that is owned by a
      different GitHub user
    \item \textbf{Origin} - The default name used by Git for a configured remote
      repository
    \item \textbf{Pull Request} - A GitHub feature which requests that changes
      from a fork be incorporated into the original repository.
  \end{itemize}
\end{bluebox}

\begin{bluebox}[title=Git Command Cheat Sheet]
  \begin{itemize}
    \itemsep\itempad
  \item \textbf{git remote add origin REPOSITORY\_URL} - Configure a local
    repository with a remote with the label 'origin'
  \item \textbf{git push} - Synchronise changes in the current local branch to
    its upstream branch
  \item \textbf{git pull} - Synchronise changes in the upstream branch to the
    current local one
  \item \textbf{git clone REPOSITORY\_URL} - Create a new local repository that
    is a copy of remote one
  \end{itemize}
\end{bluebox}


\furtherhelp

\end{document}
